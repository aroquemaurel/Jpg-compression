\documentclass[a4paper, 11pt]{article}

\usepackage{xcolor}
\input{/home/aroquemaurel/cours/includesLaTeX/couleurs.tex}
\usepackage{lmodern}
\usepackage[utf8]{inputenc}
\usepackage[T1]{fontenc}
\usepackage[francais]{babel}
\usepackage[top=1.7cm, bottom=1.7cm, left=2.5cm, right=2.5cm]{geometry}
\usepackage{verbatim}
\usepackage{tikz} %Vectoriel
\usepackage{listings}
\usepackage{fancyhdr}
\usepackage{multido}
\usepackage{amssymb}
\usepackage{multicol}
\usepackage{float}
\usepackage[urlbordercolor={1 1 1}, linkbordercolor={1 1 1}, linkcolor=vert1, urlcolor=bleu, colorlinks=true]{hyperref}

\newcommand{\titre}{Compression d'image en niveaux de gris}
\newcommand{\numero}{1}
\newcommand{\typeDoc}{DM}
\newcommand{\module}{Outils Informatiques pour le Multimédia}
\newcommand{\sigle}{OIM}
\newcommand{\semestre}{7}

\input{/home/aroquemaurel/cours/includesLaTeX/l3/tddm.tex}
\input{/home/aroquemaurel/cours/includesLaTeX/listings.tex} %prise en charge du langage C 
\input{/home/aroquemaurel/cours/includesLaTeX/l3/remarquesExempleAttention.tex}
\input{/home/aroquemaurel/cours/includesLaTeX/polices.tex}
\input{/home/aroquemaurel/cours/includesLaTeX/affichageChapitre.tex}
\makeatother
\begin{document}
	\maketitle
	\section{Compilation exécution et tests}
	L'archive qui à été déposée sur Moodle est organisé comme suit : 
	\begin{description}
		\item[\texttt{Report\_deRoquemaurelAntoine\_G1.1.pdf}] Le rapport que vous êtes en train de consulter
		\item[\texttt{Compressor/}] Programme de compression contenant les fichiers détaillés section \ref{files}. 
	\end{description}
	\subsection{Fichier sources}	
	\begin{itemize}
		\item \texttt{block.c} : Fonctions et structures de données concernant les blocks.
		\item \texttt{compressor.c} : Fonctions de compression.
		\item \texttt{decompressor.c} : Fonctions de décompression.
		\item \texttt{main.c} : Fichier principal gérant les arguments
		\item \texttt{ZIterator.c} : Itérateur effectuant un parcours en Z.
		\item \texttt{blockiterator.c} : Itérateur permettant de parcourir une image block par block.
		\item \texttt{dct-idct.c} : Fonctions appliquant la dct et son inverse
		\item \texttt{image.c} : Fonctions et structure de données concernant les images
		\item \texttt{util.c} : Fonctions utiles
		\item \texttt{Makefile} : Fichier Makefile
	\end{itemize}

	\subsection{Fichiers de tests}
	L'exécution des tests s'effectue de la même manière que l'archive fourni : 
	\begin{lstlisting}[language=Bash]
aroquemaurel@aokiji < master > : ~/projets/c/oim/jpg-compression/report
[0] % time make tests 
dct           [OK]
quantify      [OK]
vectorize     [OK]
compression   [OK]
decompression [OK]
make tests  0,38s user 0,07s system 85% cpu 0,532 total
	\end{lstlisting}
	\label{files}
	\section{Organisation des modules}
	\section{Compression}
	\section{Décompression}
	\section{Optimisations}
	\subsection{Inverse de la dct}
	\subsection{Parallélisme avec \texttt{openmp}}

\end{document}
